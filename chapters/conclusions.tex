In conclusion, the development and evaluation of the sentiment analysis tool has provided significant insight into harnessing natural language processing techniques for social media sentiment analysis. Through leveraging deep-learning models like BERT, the tool has demonstrated great success in the sentiment classification of social media posts.

Throughout the project, several critical insights were gained regarding model training, data management, and graphical user interface design. Various challenges such as over-fitting, resource constraints, API limitations, and time constraints were addressed, proving great adaptability and problem-solving skills.

Despite the success achieved in multiple areas, it is important to acknowledge the improvements that must be made to ensure an application suitable for public release. Time constraints were the main culprit for preventing the implementation of certain features, highlighting the need for better planning and a more robust development process.

Looking ahead, the project has several avenues for further exploration. Future versions of the tool could benefit by integrating additional pre-trained models, expanding support into different social medias or review sites, or implementing additional analytics features. These improvements should be paired with a robust development process to ensure the achievability of each feature.

Overall, this project has been a great learning experience, allowing the exploration of natural language processing and sentiment analysis, as well as deep-learning models such as BERT. It has laid the foundation for continued research and development within this area of computer science.